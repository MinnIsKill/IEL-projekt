\section{Příklad 1}
% Jako parametr zadejte skupinu (A-H)
\prvniZadani{E}



\begin{multicols}{2}
\let\clearpage\relax

\noindent $R_{34} = \frac{R_3*R_4}{R_3+R_4} = \frac{100*340}{100+340} = \SI{77,273}{\ohm} \\
R_{234} = R_2+R_{34} = 660+77,273 = \SI{737,273}{\ohm}$ \\ \\
Dostanu obvod: \hspace*{0.5cm} \vspace*{-0.82cm} $\xrightarrow{\hspace*{4.5cm}}$ \\
\vspace*{0.4cm} \\ \\
Pro následující výpočet rezistorů mezi uzly označenými jako A, B a C budu muset použít převod trojúhelník $\rightarrow$ hvězda. \\
$R_{A} = \frac{R_1*R_{234}}{R_1+R_{234}+R_5} = \frac{485*737,273}{485+737,273+575} = \SI{198,96}{\ohm}\\
R_{B} = \frac{R_1*R_5}{R_1+R_{234}+R_5} = \frac{485*575}{485+737,273+575} = \SI{155,166}{\ohm}\\
R_{C} = \frac{R_{234}*R_5}{R_1+R_{234}+R_5} = \frac{737,273*575}{485+737,273+575} = \SI{235,86}{\ohm}$ \\ \\
Dostanu obvod: \hspace*{0.5cm} \vspace*{-0.82cm} $\xrightarrow{\hspace*{4.5cm}}$ \\
\vspace*{0.4cm} \\ \\
Po tomto zjednodušení je již vidno, že mohu lehce dopočítat $R_{EKV}$.

\centering
\columnbreak

\vspace*{-0.9cm}

%%% 1. obvod %%%
\ctikzset{bipoles/length=1cm, bipoles/thickness = 1}
\begin{circuitikz}[scale=0.75, line width = 0.75pt]
\draw
(0,4) to[american voltage source, v^>=$U_1$] (0,2)
to [american voltage source, v^>=$U_2$] (0,0)
(0,4) -- node[circle, fill, inner sep=1.5pt, at end, label={[font=\footnotesize, text = red]above:A}][red]{} (2,4)
to [resistor, l= $R_1$] node[circle, fill, inner sep = 1.5pt, at end, label={[font=\footnotesize, text = red]above:B}][red]{} (6,4)
(2,4) -- (2,2)
to [resistor, l= $R_{234}$] node[circle, fill, inner sep = 1.5pt, at end, label={[font=\footnotesize, text = red]below:C}][red]{} (6,2)
(6,4) to [resistor, l= $R_5$]  (6,2)
to [resistor, l= $R_6$] (10,2)
(6,4) -- (10,4) to [resistor, l= $R_7$] node[circle, fill, inner sep = 1.5pt, at end]{} (10,2)
-- (10,0) to [resistor, l_= $R_8$] (0,0)
;
\end{circuitikz}

\vspace*{0.4cm}

%%% 2. obvod %%%
\hspace*{-0.6cm}
\ctikzset{bipoles/length=1cm, bipoles/thickness = 1}
\begin{circuitikz}[scale=0.75, line width = 0.75pt]
\draw
(0,3) to[american voltage source, v^>=$U_1$] (0,1.5)
to [american voltage source, v^>=$U_2$] (0,0)
(0,3) to node[circle, draw = black, fill = white, inner sep=1.5pt, midway, label={[font=\footnotesize]above:A}]{} (1,3)
to [resistor, l= $R_A$] node[circle, fill, inner sep = 1.5pt, at end]{} (3,3)
to [resistor, l= $R_B$] node[circle, draw=black, fill=white, inner sep = 1.5pt, at end,  label={[font=\footnotesize]above:B}]{} (5,4)
(3,3) to [resistor, l= $R_C$] node[circle, draw=black, fill=white, inner sep = 1.5pt, at end, label={[font=\footnotesize]below:C}]{} (5,2)
(5,4) to [resistor, l= $R_7$] (10,4)
(5,2) to [resistor, l= $R_6$] (10,2)
(10,4) -- (10,2)
-- (10,0) to [resistor, l= $R_8$] (0,0)
;
\end{circuitikz}
\end{multicols}
\let\clearpage\relax

\noindent $R_{B7} = R_B+R_7 = 155,166+255 = \SI{410,166}{\ohm},\hspace{0.5cm}     R_{C6} = R_C+R_6 = 235,86+815 = \SI{1050,86}{\ohm} \\ R_{B7C6} = \frac{R_{B7}*R_{C6}}{R_{B7}+R_{C6}} = \frac{410,166*1050,86}{410,166+1050,86} = \SI{295,0167}{\ohm}$ \\

\noindent Pak: \; $R_{EKV} = R_A+R_{B7C6}+R_8 = 198,96+295,0167+225 = \underline{\underline{\SI{718,9767}{\ohm}}} \\
\hspace*{1.1cm} U = U_1+U_2 = 115+55 =\underline{\underline{\SI{170}{\volt}}}, \hspace{0.5cm}     I = \frac{U}{R_{EKV}} = \frac{170}{718,9767} = \SI{0,236447}{\ampere} = \underline{\underline{\SI{236,447}{\milli\ampere}}}$ \\

\noindent Abychom mohli vypočítat $U_{R6}$, potřebujeme vědět proud, který rezistorem protéká, což bude stejný proud který protéká prvkem $R_{C6}$ (z 1. kirchhoffova zákona). Tudíž: $U_{R6} = I_{RC6}*R_6$. Pro výpočet $I_{RC6}$ je zase zapotřebí znát $U_{RC6}$, nebo $U_{RB7C6}$ (2. kirch. z.). \\

\noindent $U_{RB7C6} = I*R_{B7C6} = 0,236447*295,0167 = \SI{69,756}{\volt} \\
I_{RC6} = \frac{U_{RB7C6}}{R_{C6}} = \frac{69,756}{1050,86} = \SI{0,066379}{\ampere} = \underline{\underline{\SI{66,379}{\milli\ampere}}} = I_{R6}$ \\

\noindent A nakonec $U_{R6} = I_{R6}*R_6 = 66,379*815 = \underline{\underline{\SI{54,099}{\ohm}}}$

\clearpage
