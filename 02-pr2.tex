\section{Příklad 2}
% Jako parametr zadejte skupinu (A-H)
\druhyZadani{A}

\vspace*{1cm}

\begin{multicols}{2}
\let\clearpage\relax

\noindent $R_{45} = R_4+R_5 = 210+530 = \SI{740}{\ohm}$ \\
\noindent Z Theveninova teorému: $I_{R3} = \frac{u_i}{R_3+R_i}$. \\
Pro výpočet $R_i$ překreslím obvod bez $R_3$, uzel nad ním si určím jako A a dolní uzel jako B, napěťový zdroj nahradím zkratem a nakonec dopočítám $R_i$. \\ \\
Dostanu obvod: \hspace*{0.5cm} \vspace*{-0.82cm} $\xrightarrow{\hspace*{4.5cm}}$ \\
\vspace*{0.4cm} \\ \\
\noindent $R_{12} = \frac{R_1*R_2}{R_1+R_2} = \frac{100*525}{100+525} = \SI{84}{\ohm} \\
R_{456} = \frac{R_{45}*R_6}{R_{45}+R_6} = \frac{740*100}{740+100} = \SI{88,0952}{\ohm} \\
R_i = R_{456}+R_{12} = 88,0952+84 = \SI{172,0952}{\ohm}$ \\ \\
\noindent Pro výpočet $U_i$ využiju napěťový dělič (jelikož $R_6$ a $R_{45}$ jsou ve dvou větvích, na kterých je stejné napětí). \\
$U_{R6} = U*\frac{R_6}{R_2+R_6} = 50*\frac{100}{525+100} = \SI{8}{\volt} \\
U_{R45} = U*\frac{R_{45}}{R_1+R_{45}} = 50*\frac{740}{100+740} = \SI{44,0476}{\volt} \\
U_i = \lvert U_{R6}-U_{R45} \rvert = \lvert 8-44,0476 \rvert = \SI{36,0476}{\volt}$ \\ \\
Dostanu obvod: \hspace*{0.5cm} \vspace*{-0.82cm} $\xrightarrow{\hspace*{4.5cm}}$ \\
\vspace*{0.4cm} \\ \\




\centering
\columnbreak

\vspace*{0.3cm}

%%% 1. obvod %%%
\ctikzset{bipoles/length=1cm, bipoles/thickness = 1}
\begin{circuitikz}[scale=0.75, line width = 0.75pt]
\draw
(0,0) -- (0,3)
to node[circle, fill, inner sep=1.5pt, at end]{} (1,3)
(1,4) to [resistor, l= $R_1$] node[circle, fill, inner sep = 1.5pt, at end, label={[font=\footnotesize, text = red]above:A}][red]{} (4,4)
(1,4) -- (1,2)
to [resistor, l= $R_2$] node[circle, fill, inner sep = 1.5pt, at end, label={[font=\footnotesize, text = red]below:B}][red]{} (4,2)
to [resistor, l= $R_6$] (7,2)
(4,4) to [resistor, l= $R_{45}$] (7,4) -- (7,2)
(7,3) to node[circle, fill, inner sep=1.5pt, at start]{} (8,3) -- (8,0) -- (0,0)
;
\draw[green] (4,4) to [open, color= green, v^=$u_i$][green] (4,2)
;
\end{circuitikz}

\vspace*{3.2cm}

%%% 2. obvod %%%
\hspace*{-0.6cm}
\ctikzset{bipoles/length=1cm, bipoles/thickness = 1}
\begin{circuitikz}[scale=0.75, line width = 0.75pt]
\draw
(0,3) to[american voltage source, v_>=$U_i$] (0,0)
(0,3) to [resistor, l= $R_i$] node[circle, draw=black, fill=white, inner sep = 1.5pt, at end, label={[font=\footnotesize]above:A}]{} (3,3)
(3,0) to node[circle, draw=black, fill=white, inner sep = 1.5pt, at start, label={[font=\footnotesize]below:B}]{} (0,0)
;
\draw[dashed] (3,3) to (3,2);
\draw (3,2) to [resistor,thick, l= $R_3$] (3,1);
\draw[dashed] (3,1) to (3,0)
;
\end{circuitikz}
\end{multicols}
\let\clearpage\relax

\noindent Poté už stačí pouze dosadit do vzorce: $I_{R3} = \frac{u_i}{R_3+R_i} = \frac{36,0476}{620+172,0952} = \SI{0,045509}{\ampere} = \underline{\underline{\SI{45,509}{\milli\ampere}}}$ \\
Z toho pak: $U_{R3} = I_{R3}*R_3 = 0,045509*620 = \underline{\underline{\SI{28,216}{\volt}}}$ \\

\clearpage
